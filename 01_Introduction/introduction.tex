%%%%%%%%%%%%%%%%%%%%%%%%%%%%%%%%%%%%%%%%%%%%%%%%%%%%%%%%%%%%%%%%%%%%%%%%%%%%%%%%
%2345678901234567890123456789012345678901234567890123456789012345678901234567890
%        1         2         3         4         5         6         7         8
% THESIS INTRODUCTION

\chapter{Introduction}
\label{chap:introduction}
\ifpdf
    \graphicspath{{Introduction/Figures/PNG/}{Introduction/Figures/PDF/}{Introduction/Figures/}}
\else
    \graphicspath{{Introduction/Figures/EPS/}{Introduction/Figures/}}
\fi

Water is one of the most ubiquitous, and universally experienced, substances on this planet. It is rather counter-intuitive, then, how much remains to be understood about water. The situation is even more precarious when more exotic forms of water are concerned, for instance water at extremely high pressures or within nanoscale confinement. The latter of these two is the primary focus of this study. Specifically, how does the confinement of single water molecules in the naturally occurring nanocages of the beryl crystal affect its structural and electronic properties? 



\section{Context of and Motivations for Study}
\label{motivations}

Confined water has been the subject of numerous studies, and is believed to be broadly applicable, including in mineralogy, geology, climatology, meteorology and cosmology; but the area of greatest potential impact is in biology [\textbf{list of citations}]. Biological systems often create naturally occurring channels wherein individual water molecules traverse the system or interfaces where single molecules are participating in hydrophilic/hydrophobic interactions, transcription and replication processes. If these biological processes are to ever be fully understood, the effects of confinement need to be fully understood. Unfortunately, attempting to study confined water in biological is extremely difficult, for any interpretation of the experimental results is obfuscated by the systems’ complexity. Studies of simpler systems, then, are needed to gain insight into the fundamental mechanisms behind these biological processes.  

Nature has provided an ideal “scaffolding” with which to perform such a study, namely with a family of crystals of which beryl is a member. Like all crystals, the periodic nature of these structures provides the idyllic, simple geometry that biological systems often lack. Additionally, beryl crystals contain naturally occurring cavities, or nanocages, with dimensions perfectly suited for trapping single water molecules. When beryl crystals are grown in water environments, single water molecules are trapped within said nanocages. Moreover, these nanocages are spaced at such a distance that the hydrogen bonds that dominate the bulk characteristics of water are interrupted, and the dipole interactions are able to govern.
 
Some explorations of this system have already been conducted, both theoretically and experimentally, which provide the foundation for this study. From optical spectroscopy experiments, two different types of confined water have been identified [\textbf{nature paper}], as well as the different vibrational modes accessible to said water molecules [\textbf{vibrational spectra paper}]. The system has also been probed theoretically, where it is believed to be a good candidate for a quantum electric dipole system [\textbf{qed paper}]. 

These studies have provided a macroscopic-esque view of the system, but want for a low-level investigation of the dipole-dipole dynamics still exists. Due to the time- and length scales of this system, such investigations should be carried out computationally.

\section{Objectives and Contributions}
\label{objectives}

The primary objective of this project is to provide foundational knowledge so that larger computational studies can be performed. To that end, assumptions about the coupling between the crystal framework and the extra-framework water molecules need to be tested, the geometry of the water molecules needs to be determined with respect to fill factor and relative position to other water molecules and the framework, a potential energy map for the high-symmetry point needs to be calculated over the same fill and relative position parameter space, and the molecular dipole needs to be determined in all of these cases as well. Finally, the contributions of the crystal framework and of the water-water subsystems need to be disentangled.

As previously mentioned, beryl is a single example of a class of crystals in which such a study can be performed. As such, the "path of least computational resources" will be identified, so that future studies of these other crystals can be performed in a systematic, cost-effective way. 

\section{Overview of the Thesis}
\label{overview}

The remainder of this thesis is broken down into five chapters. First, the necessary background information is presented, including theory, system overview, and details of the calculations. The following three chapters represent the bulk of this work and present the structural optimization, potential map calculations, and partial charge analysis. Each chapter is further subdivided into separate experiments, wherein the motivation, procedure, results, and analyses are presented. At the end of each chapter, a brief summary of all of the results is provided as well as an outlook for future work. The concluding chapter provides a high-level summary of the work in its entirety as well as a general outlook for the computational investigation of this system.