%%%%%%%%%%%%%%%%%%%%%%%%%%%%%%%%%%%%%%%%%%%%%%%%%%%%%%%%%%%%%%%%%%%%%%%%%%%%%%%%
%2345678901234567890123456789012345678901234567890123456789012345678901234567890
%        1         2         3         4         5         6         7         8
% THESIS INTRODUCTION

\chapter{Introduction}
\label{chap:introduction}


Water is one of the most ubiquitous, and universally experienced, substances on this planet. It is rather counter-intuitive, then, how relatively little is known about it. The situation is even more precarious when exotic forms of water are concerned, for instance water at extremely high pressures or within nanoscale confinement\embdashtext the latter of these two being the primary focus of this study. %Specifically, how does the confinement of single water molecules in the naturally occurring nanocages of the beryl crystal affect its structural and electronic properties? 



\section{Context of and Motivations for Study}
\label{motivations}

Confined water has been the subject of numerous studies, and is believed to be broadly applicable, including in mineralogy, geology, climatology, meteorology and cosmology; but the area of greatest potential impact is in biology [\textbf{list of citations}]. Biological systems often create naturally occurring channels, wherein individual water molecules traverse the system, or interfaces, at which single molecules are participating in hydrophilic/hydrophobic interactions, transcription, and replication processes. If fundamental understanding of these biological processes is ever to be achieved, the effects of water molecule confinement need to be fully understood. Unfortunately, attempting to study confined water in biological systems is extremely difficult, for any interpretation of the experimental results is obfuscated by the system's complexity. Studies of simpler systems, then, are desired to gain greater insight into the fundamental mechanisms behind these biological processes.  

Nature has provided an ideal \textit{scaffolding} with which to perform such a study, namely with a family of crystals of which beryl is a member. Like all crystals, the periodic nature of these structures provides the idyllic, simple geometry that biological systems often lack. Additionally, beryl crystals contain naturally occurring cavities, or nanocages, with dimensions perfectly suited for encapsulating single water molecules, and when beryl crystals are hydrothermally grown, these nanocages become occupied by a single water molecule. Moreover, these nanocages are spaced at such a distance that the hydrogen bonds that normally dominate the interactions are interrupted, and the dipole interactions are able to govern. (More details provided in Chapter 2.)
 
Some explorations of this system have already been conducted, both theoretically and experimentally, which provide the foundation for this study. As discussed in further detail in the next chapter, two different types of confined water have been identified using optical spectroscopy experiments [\textbf{nature paper}]. Additionally, the different vibrational modes accessible to said water molecules have been modeled and measured experimentally [\textbf{vibrational spectra paper}]. The system has also been probed theoretically, where it is believed to be a good candidate for a quantum electric dipole system and has shown to demonstrate evidence of ferroelectric behavior [\textbf{qed paper}]. 

All of these studies provide a macroscopic-esque view of the system, but want for a low-level investigation of the dipole-dipole dynamics still exists. Due to the time- and length scales of this system, such investigations should be carried out computationally. While quantum mechanical molecule dynamics is possible, it would be cheaper to use a coarse-grain model for nanoconfined water. However, there exist no physical justification for choosing a suitable model from the zoo of available water models, for these models were developed to capture bulk properties. This work aims to fill that void and provide said physical justification. 

\section{Objectives and Contributions}
\label{objectives}

The primary objective of this work is to determine whether or not it makes sense to attempt to parameterize a coarse-grained force field for nanoconfined water. Specifically, how do the likely parameters change, if at all, with respect to the confined-water parameter space? For instance, do the OH bond length $r$ and OHO angle $\theta$ distributions behave in a way that make average descriptions meaningful? How does one assign partial charges to the confined water ions? Is it feasible to use a Lennard-Jones potential? What cut-off distance should be used for the electrostatic interactions? Assuming such descriptions are meaningful, what parameterization procedure should be followed to minimize computational cost? How strongly do the crystal framework interactions influence these parameters? As previously mentioned, beryl is one example of a broader class of crystals, more of which will want to be studied to generalize the understanding of confined water dynamics.

As a secondary objective, an attempt is made to answer the following: What does the potential map look like with respect to rotation about the high-symmetry point? The motivation for answering this question is two-fold. First, the potential map is a suitable observable, against which a force field can be tuned and verified. Second, as discussed in Chapter \ref{ch:pot_map}, there exist discrepancy in the literature as to what this potential map looks like. Through explicit investigation, this controversy can be resolved.



\section{Overview of the Thesis}
\label{overview}

The remainder of this thesis is broken down into five chapters. First, the necessary background information is presented, including theory, system overview, and details of the calculations. The following three chapters represent the bulk of this work and present the structural optimization, potential map calculations, and partial charge analysis. Each chapter is further subdivided into separate experiments, wherein the motivation, procedure, results, and analyses are presented. At the end of each chapter, a brief summary of all of the results is provided as well as an outlook for future work. The concluding chapter provides a high-level summary of the work in its entirety as well as a general outlook for the further investigation of this system.